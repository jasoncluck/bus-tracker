\documentclass[12pt]{article}

\usepackage{graphicx}
\usepackage{enumerate}
\usepackage{listings}
\usepackage{multicol}

%\linespread{1.6}

\title{Live Map of the DC Bus System}
\author{Ian Will, Jason Cluck}
\date{}

\begin{document}
\maketitle

\begin{abstract}
The Washington Metropolitan Area Transit Authority (WMATA)  publishes an API providing access to live bus positions and route details.  There are a few iOS, Android, and Web applications that use this data to help make transit riding more pleasant.  For example, the NextBus application shows the expected arrival time of the next bus for a given route and stop.

However, all current applications have usability problems that limit their usefullness.  None shows all bus routes simultaneously on a map, which limits a transit rider's ability to discover alternate routes.  None shows live animations of current bus positions.  A combination of these two points of information would be very helpful for a bus rider--if the desired bus requires a long wait, a short walk to a nearby station on another route with an approaching bus may provide a better alternative.

We will use the WMATA developer API to retrieve route details and bus positions from WMATA's servers at regular intervals.  We will then use the Google Maps API to display the route data, updated in real time.  The WMATA developer API places restrictions on frequency of data access--five times per second or 10,000 times per day maximum.  Those limits prohibit accessing the data every time our web page is loaded (were it to be widely used); therefore our application will poll the WMATA servers roughly once every 5 seconds and serve the latest snapshot from our separate web server.  We will use the Ruby on Rails web framework to build a web server that maintains a database of route and position data, updates from WMATA's servers periodically, and displays the data on a map efficiently.  The application will be deployed using the Heroku cloud application platform.

\end{abstract}



\end{document}